\documentclass[14pt]{extreport}

\usepackage[utf8]{inputenc}
\usepackage[italian,english]{babel}
\usepackage{graphicx}
\usepackage{cite}
\usepackage{amsmath}
\usepackage[table,xcdraw]{xcolor}
\usepackage[italian]{minitoc}% per gli accenti
\usepackage{fancybox}
\usepackage{verbatim}
\usepackage{url}
\usepackage{color}
\usepackage{listings}
\usepackage{makeidx}
\usepackage{comment}
\usepackage{fancyhdr}
\pagestyle{fancy}
\fancyhf{}
\rhead{Cap.\thechapter}
\fancyhead[L]{\rightmark}


\lstset{frame=tb,
	language=Java,
	numbers=left,
	keywordstyle=\color{blue},
	alsoletter={.}
}
\graphicspath{ {./Figure/} }

\usepackage{titlesec}

\titleformat{\chapter}{\normalfont\huge\bf}{\thechapter.}{20pt}{\huge\bf}
\titleformat{\section}{\normalfont}{\thesection.}{16pt}{\normalfont}


\begin{document}
\selectlanguage{italian}
\begin{titlepage}
\begin{center}
	\begin{figure}
    	\includegraphics[width=1.5cm, height=1.5cm]{unisa.png}
    	\centering
    \end{figure}
	{\Large Università degli Studi di Salerno}\\[0.2truecm]
	{\large Dipartimento di Informatica\\Corso di Laurea Magistrale in Informatica}\\
	\hrulefill
	\vfill
	{\large Tesi di Laurea Magistrale in Informatica}\\[0.2truecm]
	%{\Large Informatica}\\
	\vfill\vfill
	{\LARGE {\bf Titolo}}
	
	\vfill\vfill
	
	
	\ \ \ \ \ \ \ {\bf Docente} \hfill {\bf Candidato}\ \ \\
	Prof.ssa \textbf{Delfina Malandrino} \hfill  \textbf{Francesco Vicidomini}
	\centerline{\hfill Matricola: 051210533}
	
	
	\vfill
	\hrulefill 
	\begin{center} Anno Accademico 2019-2020 \end{center}
	
\end{center}
\end{titlepage}

\setcounter{page}{1} 		
\pagenumbering{roman} 

\newpage
	

\tableofcontents
\listoffigures %elenco figure
\listoftables %elenco tabelle
%%%%%%%%%%%%%%%%%%%%%%%%%%%%




%%%%%%%%%%%%%%%%%%%%%%%%%%%%
\chapter*{Abstract}
 
Gli Online Social Network(OSN) sono tra le applicazioni che hanno maggiore utenza e posseggono gran parte dei dati personali dei loro utenti.\newline
La gestione dei dati personali da parte di questi colossi dell'IT viene regolata attraverso apposite leggi, ma spesso sono gli utenti stessi a pubblicare involontariamente delle informazioni sensibili che possono esporli a dei rischi. Ad esempio abbiamo sentito spesso parlare di furti in appartamento fatti proprio perché i ladri avevano visto sui social delle vittime che queste erano partite per una vacanza.\newline
L'obiettivo di questa tesi è aggiungere al plugin di Chrome Knoxly la possibilità di riconoscere dati sensibili in base al contesto in cui sono stati scritti e in base a quanto l'utente reputi sensibili questi dati. Per raggiungere questo obiettivo è stata sviluppata e installata all'interno di Knoxly una intelligenza artificiale in grado di riconoscere il dato sensibile estrapolandolo dalla semantica della frase in cui viene menzionato, in seguito l'utente indicherà quanto questo dato sia sensibile per egli. Ad esempio se il proprietario di una pasticceria tweetta "Ho una pasticceria in via Roma" l'indirizzo viena riconoscito come un dato sensibile ma per il pasticciere questo non lo è perchè sta usando quel tweet per farsi pubblicità quindi egli comunicherà al sistema che per lui quel dato non è sensibile e il sistema aggiornerà la sua IA.


%%%%%%%%%%%%%%%%%%%%%%%%%%%%

%%%%%%%%%%%%%%%%%%%%%%%%%%%%
\chapter{Titolo capitolo}
\setcounter{page}{1} 		
\pagenumbering{arabic}
\section{Contesto}
Da una decina d'anni a questa parte gli OSN sono sempre più presenti nella vita di milioni di persone; siamo arrivati al punto che una singola persona possiede anche più tipi di social network (o addirittura account multipli) e su ognuno di questi, spesso sulla base del tipo di comunicazione che permette, pubblica contenuti diversi. Alla base dei social network c'è il concetto di \quotes{\textit{condivisione}} fino a quando un utente si limita a condividere su queste piattaforme contenuti propri ed innocui questo non causa alcun tipo di complicazione, anzi, è un modo per comunicare con il resto del mondo. I problemi sorgono quando sui social vengono condivisi contenuti che possono violare la privacy propria o altrui. 

In particolare, alcuni studi dimostrano che gli utenti a volte rivelano troppe informazioni o divulgano involontariamente messaggi che minano la loro privacy, specialmente quando sono negligenti, emotivi o inconsapevoli dei rischi~\cite{looseTweets, studyFb, readMyTwitter}.
Per la maggior parte degli utenti, i loro post sono pensati solo per essere condivisi con amici / follower, ma il pubblico degli OSN è significativamente più grande delle aspettative degli utenti, tra cui inserzionisti, recruiter, robot dei motori di ricerca, ecc.
Spesso queste informazioni pubblicate sugli OSN possono contenere informazioni sensibili e/o personali proprie o di terzi. 

Quando gli utenti di un OSN si registrano ad esso questi stanno affidando alla piattaforma alla quale si iscrivono la loro anagrafica e molte altre informazioni che possono contribuire alla identificazione univoca di se stessi. A gestori degli OSN viene affidato il compito di manutenere e di tenere al sicuro i dati che l'utente gli affida. Purtroppo sono stati riportati dei casi in cui delle falle di sicurezza presenti in alcuni OSN abbiano portato al furto di molti account (e ai dati ad esso connessi).

Pertanto, c'è la necessità di poter identificare contenuti online potenzialmente sensibili, in modo che gli utenti possano essere avvisati di tali minacce ed agire di conseguenza prima che il contenuto venga pubblicato in rete.

%A tal proposito è stato dimostrato, che su una sessione di navigazione che comprendeva un insieme di azioni tipiche di un utente (accesso a social network, ricerche, posting di commenti, ecc), Google Analytics raccoglieva circa l’87\% dei dati personali e sensibili disponibili~\cite{MalandrinoScarano}.



\subsection{Alcuni casi di furto di informazioni}
\label{ssec:exampleLeak}
\textbf{1. AOL search data leak}\cite{aolDataLeak}\newline
Nel 2006 la compagnia internet AOL pubblica un grande numero di ricerche fatte dagli utenti. Le ricerche pubblicate da AOL erano anonime ma, molte di queste, contenevano informazioni personali sensibili.

Partendo da queste ricerche, il \textit{New York Times} è stato in grado di risalire a molte delle persone che avevano effettuato quelle ricerche.\newline
\textbf{2. Myspace account leakage}\cite{myspacebreach}\newline
Nel 2016 sono stati trovati 360 milioni di account myspace in vendita sul deep web. I dati comprendevamo mail, password, username di una parte degli utenti create prima del giugno 2013.\newline
\textbf{3. Cambridge Analytica}\cite{cambridge}\newline
Nel 2018 viene rivelato che la società di consulenza politica Cambridge Analytica aveva raccolto i dati di milioni di utenti Facebook e li ha usati per scopi di propaganda politica.

I dati raccolti da Cambridge Analytica sono stati utilizzati per la campagna elettorale di Donald Trump, Brexit, e le elezioni messicane del 2018.\newline
\textbf{4. Caso Renzi Formigli}\cite{formigli}\newline
In questo caso non c'è un colosso della IT che per una falla di sicurezza perde le informazioni sensibili dei sui utenti ma una persona che rivela in maniera indiretta dati sensibili di una terza.

Il giornalista e conduttore televisiovo Corrado Formigli intervista il senatore della Repubblica Matteo Renzi in merito al prestito che gli era stato fatto da un suo amico per l’acquisto di una villa.


Durante la trasmissione il conduttore mostra le foto della villa, poche ore dopo il termine della trasmissione su Facebook compaiono post che parlano e mostrano la casa del giornalista.

\section{Privacy leakage}
\label{sec:privacyLeakage}
Nella sezione \ref{ssec:exampleLeak} sono stati illustrati alcuni esempi di \textbf{privacy leakage}, ovvero, contenuti sensibili o inappropriati che gli utenti stessi, inconsapevolmente o involontariamente, divulgano durante l'utilizzo dei social o altri servizi (es. e-mail).I contenuti possono riguardare la propria vita privata o quella altrui.

\section{Norme per tutelare la privacy degli utenti}
Per migliorare e regolamentare la gestione dei dati sensibili e personali sul web la Commissione Europea ha iniziato ad aggiornare le regole in ambito privacy dal 2016, concentrandosi sul controllo della diffusione dei dati e sul conoscere da \textit{chi} e \textit{come}, questi vengano trattati.

La normativa europea GDPR(General Data Protection Regulation)\footnote{https://gdpr-info.eu/} definisce come \textit{dati personali} le informazioni che identificano o rendono identificabile, direttamente o indirettamente, una persona fisica e che possono fornire informazioni sulle sue caratteristiche, le sue abitudini, il suo stile di vita le sue relazioni personali, il suo stato di salute, la sua situazione economica, ecc.

Il GDPR quindi riduce il rischio di uso improprio e violazioni della privacy,
causate intenzionalmente o per mancanza di consapevolezza.

Inoltre la Convenzione 108\footnote{https://www.coe.int/en/web/data-protection/convention108/modernised} e il Regolamento Europeo prevedono che i dati devono
essere conservati per un periodo di tempo limitato, e in particolare non
oltre il tempo necessario per raggiungere lo scopo alla base del trattamento.
Nel caso in cui un titolare del trattamento volesse mantenerli per un periodo
superiore, deve procedere alla loro anonimizzazione.

\section{Cos'è Knoxly}
\label{sec:whoisKnoxly}
Knoxly (Figura \ref{fig:logoKnoxly}) si prefigge come obiettivo quello di aiutare gli utenti ad avere una maggiore consapevolezza
dei dati che pubblicano/condividono sul Web , dato che il principale rischio per la privacy online è la divulgazione di
informazioni personali e sensibili degli stessi utenti.\newline
\begin{figure}[h]
    \centering
    \includegraphics[scale=0.25]{Figure/logoKnoxly.png}
    \caption{logo di Knoxly}
    \label{fig:logoKnoxly}
\end{figure}
\FloatBarrier
Spesso gli utenti degli OSN divulgano anche senza la piena consapevolezza dati sensibili e/o personali proprio o di terzi. A tal fine, Knoxly (1) individua quali dati sensibili e personali sono presenti in un messaggio di testo che l’utente è intenzionato a pubblicare/condividere sul Web, (2) li mostra attraverso un’ interfaccia grafica user-friendly e (3) lascia all’utente decidere quali o quante informazioni sensibili e personali è disposto a condividere con la divulgazione di tale messaggio.

A differenza dei convenzionali meccanismi di protezione della privacy sui dati o sui OSN, che si focalizzano principalmente sulla protezione dell’identità o di attributi privati \cite{collective-data, diff-privacy, urwho, inferr-privacy, protection-private, stalking}, Knoxly si muove verso l’idea di \quotes{\textit{privacy as having the ability to control the dissemination of sensitive information}}.

Attualmente i limiti di Knoxly sono quelli intrinsechi ad una analisi lessicale. Per questa ragione, in questo lavoro ci si vuole muovere verso metodi intelligenti che oltrepassino i metodi keyword-based, aggiungendo un \quotes{modulo semantico} in grado di capire se un testo scritto da un utente contenga o meno dati sensibili e/o personali.

\section{Il modello di minaccia}
Knoxly mira a proteggere gli utenti del Web dalla diffusione accidentale di qualsiasi contenuto inappropriato, in particolare le informazioni private o sensibili su se stessi. Si considera principalmente il rischio di una diffusione inappropriata a due tipo di pubblico: (1) follower o amici, che ricevono aggiornamenti dei post dell'utente; (2) staker esterni, che sbirciano nei post dei social network di un utente target. È probabile che entrambi conoscano l'identità offline dell'utente. Knoxly non impedisce all'utente di pubblicare i contenuti(sensibili) nè impedisce al destinatario di visualizzare i contenuti, fornendo privacy awareness. L'idea di Knoxly parte dal presupposto che dei tracciatori possono esplorare l'OSN attraverso l'interfaccia utente o raccogliere dati utilizzando un crawler automatizzato tramite l'API dell'OSN di riferimento. Infine, non considera la retrazione/cancellazione dei post precedenti, nè li analizza.

\section{Obiettivi del lavoro}
\label{sec:obiettiviCap1}
Questo lavoro si prefigge l'obiettivo di aggiungere a Knoxly un modulo intelligente in grado di poter riconoscere un dato sensibile in un testo scritto in linguaggio naturale senza affidarsi a meccanismi keyword-based, di segnalare all'utente la presenza di questo dato e di dare la possibilità all'utente di personalizzare il livello delle segnalazioni in modo tale che esso possa \quotes{imparare} a riconoscere quali sono i dati sensibili secondo lo specifico l'utente che sta utilizzando il plugin.

\subsection{I dati sensibili}
\label{ssec:sensitive_data}
Nelle precedenti sezioni si è molto discusso di dati sensibili, ma cosa sono esattamente? Secondo la normativa GDPR attualmente in vigore per dati sensibili si intendono i dati che rivelino:
\begin{itemize}
    \item l'etnia, pensiero politico, religioso o filosofico
    \item se si è membri di un sindacato
    \item dati genetici o biomedici che permettano l'identificazione univoca di un essere umano
    \item dati relativi alla salute
    \item orientamento sessuale e vita sessuale
\end{itemize}
Abbiamo scelto di utilizzare un sottoinsieme di temi di partenza per dare una conoscenza di base all'IA di Knoxly. L'IA inizialmente sarà in grado di individuare dati sensibili riguardo i seguenti topic
\begin{enumerate}
    \item orientamento politico
    \item salute
    \item lavoro
    \item viaggi
\end{enumerate}
Questi temi sono stati scelti per due motivi: (a) molti di questi temi fanno parte della normativa GDPR quindi è una legge a dirci che i dati appartenenti a quelle categorie sono sensibili e (b) molti dei temi che verranno trattati sono già stati utilizzati in vari riguardanti la privacy detection e il privacy leakage~\cite{looseTweets, MalandrinoScarano, dontTweetThis}.

\section{Struttura della tesi}
\label{sec:res_ottenuti}
Una volta descritti gli obiettivi che si pone questo lavoro di tesi (Sez.~\ref{sec:obiettiviCap1}), nei successivi capitoli verrà descritto in modo dettagliato il percorso e le scelte fatte per raggiungere tali obiettivi.

Nel capitolo \ref{ch:cap2} viene descritto il background tecnologico e teorico delle varie IA che compongono il modulo semantico di Knoxly.

Prima di poter iniziare la sperimentazione e la realizzazione del modulo semantico di Knoxly nel capitolo \ref{ch:cap3} vengono descritti i lavori principali presenti in letteratura riguardanti il tema della privacy awareness e del privacy leakage.

Dopo aver visto i principali lavori presenti in letteratura riguardo il tema della privacy awareness e del privacy leakege è stato possibile capire quali topic sono soggetti al leakage. Oltre a questo è stato possibile capire il modus operandi da applicare per la realizzazione del modulo semantico. Sulla base di questi studi nel capitolo \ref{ch:cap4} si descrive in maniera dettagliata tutto il processo di sviluppo del modulo semantico partendo dalla collezione dei dati utilizzanti (Sez. \ref{datacollection}) fino ad arrivare alla realizzazione dei 3 classificatori che compongono il modulo semantico. I classificatori realizzati sono il Topic classifier (Sez. \ref{sec:topicclass} ovvero un classificatore in grado di riconoscere il topic a cui fa riferimento una frase, il Sensitivenss classifier (Sez. \ref{sec:sensclass}) ovvero un classificatore un grado di dire se un contenuto è sensibile o meno e il Customized Sensitiveness classifier (Sez. \ref{sec:pres_sens_class}), un classficatore che in base ai feedback forniti dall'utente riuscirà a riconoscere se una frase risulta essere sensibile o meno in base alla percezione della privacy dell utente.

Una volta realizzato il modulo semantico nel capitolo \ref{ch:cap5} viene descritto come questo sia stato integrato in Knoxly e nel capitolo \ref{ch:cap6} viene effettuato uno studio quantitativo e qualitativo sulle performance di Knoxly. Nello studio quantitativo sono state misurate le perfomance sia del client che del server di Knxoly per vedere quanta RAM, tempo e CPU utilizzasse il plug-in. Nello studio qualitativo invece sono state misurate le performance di Knoxly per valutare quanto le previsione fatte dai classificatori siano corrette.

Infine nel capitolo \ref{ch:cap7} viene fatta una riflessione sui risultati ottenuti da questo lavoro e su quali possano essere i possibili sviluppi futuri del progetto Knoxly.

%%%%%%%%%%%%%%%%%%%%%%%%%%%%
\chapter{Titolo capitolo}
Per problemi di classificazione o di previsione vengono utilizzati diversi modelli di machine learning. In questa capitolo verranno descritti brevemete i metodi che verranno utilizzati.
\section{Modelli utilizzati}
\subsection{Random Forest}
Il Random Forest(RF) è un algoritmo di supervised classification che consiste in un insieme di metodi basati sul bagging\cite{Random Forest}; è stata usata l'implementazione di \textit{scikit-learn} combina gli alberi facendo la media della loro previsione probabilistica invece di lasciare che ogni albero voti per una singola classe, e supporta intrinsecamente problemi multi-classe.

Esistono diverse metriche per valutare i punteggi dei modelli di machine learning. In questo lavoro sono state utilizzate le seguenti metriche:

\begin{enumerate}
    \item \textbf{accuracy}: frazione delle predizioni corretta fatte dal nostro modello, essa è definita come segue
    \begin{equation*} accuracy = \dfrac {(tp + tn)}{(tp + tn + fp + fn)}\end{equation*} 
    dove $tp$, $fn$, $fp$ e $tn$ sono rispettivamente il numero di veri positivi, falsi negativi, falsi positivi, veri negativi
    
    \item \textbf{precision}: capacità del classificatore di non etichettare come positivo un campione che è negativo, definita come segue
    \begin{equation*} precision = \dfrac {tp}{(tp + fp)}\end{equation*} 
    dove $tp$ rappresenta il numero di veri positivi e $fp$ il numero di falsi positivi
    
    \item \textbf{recall}: capacità di un classificatore di trovare tutti i campioni positivi; definita come
    \begin{equation*} recall = \dfrac {tp}{(tp + fn)}\end{equation*} 
    dove $tp$ rappresenta il numero di veri positivi e $fn$ il numero di falsi negativi
    
\end{enumerate}{}

\section{Universal Sentence Encoder}
Universal Sentence Encoder(USE) è un encoder che trasforma del testo scritto in linguaggio naturale in un vettore di grandi dimensioni, questo encoder può essere usato per task di text classification, semantic similarity, clustering e altri task riguardanti l'ambito dell'analisi del testo scritto in linguaggio naturale.\newline
Nel seguente esempio viene mostrato il funzionamento di USE, egli prende in input un testo scritto in linguaggio naturale, codifica la frase in un array di 512 numeri reali, infine è stata misurata la similarità semantica delle frasi codificate.
\begin{figure}[h]
    \centering
    \includegraphics [scale=0.42]{Figure/use.png}
    \caption{Esempio funzionamento USE}
    \label{fig:my_label}
\end{figure}
USE ha vari tipo di encoder, quello utilizzato in questo lavoro è un encoder multilingua che supporta 16 lingue diverse(arabo, cinese-semplificato, cinese-tradizionale, francese, inglese, italiano, giapponese, coreano, olandese, polacco, portoghese, spagnolo, tailandese, turco, russo, tedesco).

\chapter{Titolo capitolo}
In letteratura il problema affrontato in questa tesi è stato toccato da diversi contributi. In questo capitolo si illustrano due articoli molto correlati al lavoro qui presentato. Uno dei primi studi che ha riguardato il livello di privacy leakage contenuto nei mesaggi di testo divulgati in rete e più nello specifico su quelli divulgati su Twitter, viene condotto da Mao et al.~\cite{looseTweets}. In questo lavoro i ricercatori hanno cercato di categorizzare la natura dei leak nelle seguenti categorie:
\begin{itemize}
    \item leak provenienti da tweet di utenti in stato di ebrezza
    \item leak provenienti da tweet che rivelano i piani vacanze degli utenti
    \item leak provenienti da tweet che rivelano le informazioni mediche degli utenti
\end{itemize}
Per categorizzare la natura dei tweet verrà realizzato un classificatore in grado di rilevare i tweet affetti da leakage e infine viene effettuato uno studio comparativo per vedere quali argomenti sensibili vengono trattati in diverse nazioni.

La metodologia utilizzata da Mao et al.~\cite{looseTweets} per raggiungere i proprio obiettivi è la seguente
Inizialmente è stato realizzato un dataset contente i tweet presi dalla stream di Twitter nel periodo che va da gennaio a settembre 2010, una volta raccolti i tweet nel dataset questo è stato filtrato eliminando dal dataset di partenza i tweet che non contenevano le parole più ricorrenti trattate nelle categorie sopracitate. 

Utilizzando naive Bayes e SVM viene relizzato un classificatore in grado di rilevare la presenza di un leak in un tweet. Grazie ad esso è stato possibile effettuare una analisi comparativa per vedere quali argomenti vengono trattati principalmente nei testi affetti da leakage. I risultati sono i seguenti: in Figura~\ref{fig:res-ebbrezza} si mostrano i principali argomenti trattati nei tweet affetti da leak. Si nota come il 25\% riguardi "comportamenti irrispettosi".
\begin{figure}[h!t]
    \centering
    \includegraphics[width=15cm]{Figure/related_work/ebbrezza.png}
    \caption{argomenti trattati nei tweet fatti da utenti in stato di ebbrezza. Dati estrapolati da\cite{looseTweets}}
    \label{fig:res-ebbrezza}
\end{figure}
\FloatBarrier

In Figura~\ref{fig:res-patologie} si mostrano le patologie di cui maggiormente si parla nei tweet affetti da leak. Notiamo come, nel campione analizzato, più del 60\% dei tweet che trattano di salute vertano su \quotes{cancro}.

\begin{figure}[h!t]
    \centering
    \includegraphics[width=15cm]{Figure/related_work/patologie.png}
    \caption{patologie contenute nei tweet che parlano di salute. Dati estrapolati da\cite{looseTweets}}
    \label{fig:res-patologie}
\end{figure}
\FloatBarrier

Infine i ricercatori hanno voluto effettuare una analisi comparativa andando a vedere quali patologie venivano trattate nei tweet riguardanti la categoria salute in tre diverse nazioni (Figura~\ref{fig:res-patologie-nazioni}). In tutte le nazioni considerate, USA, UK, Singapore la patologia più discussa è sempre il \quotes{cancro}.

\begin{figure}[h!t]
    \centering
    \includegraphics[width=15cm]{Figure/related_work/nazioni.png}
    \caption{patologie contenute nei tweet che parlano di salute. Dati estrapolati da\cite{looseTweets}}
    \label{fig:res-patologie-nazioni}
\end{figure}
\FloatBarrier


Fra le limitazioni di questo studio troviamo: (a) il modello realizzato non viene integrato in un tool, e/o messo a disposizione degli utenti finali, (b) la selezione dei testi che possono appartenere ad una categoria viene effettuata sulla base di keyword (non si tiene conto della semantica).


\begin{comment}
Al fine di tutelare i dati che gli utenti condividono con il loro passaggio online il 25 maggio del 2018 è entrata in vigore la normativa europea sulla protezione dei dati personali meglio nota come regolamento GDPR.\footnote{\url{https://eur-lex.europa.eu/legal-content/EN/TXT/HTML/?uri=CELEX:32016R0679&from=EN\#d1e40-1-1}}
Questa normativa limita il commercio dei dati ma di certo non impedisce agli utenti di pubblicare su un OSN informazioni sensibili in maniera volontaria o involontaria su se stesso o, peggio ancora, su altri. A tal proposito Mao et al. \cite{looseTweets} hanno condotto uno studio quantitativo su Twitter, lo studio ha rivelato che nei tweet sono presenti molti dati personali non solo riguardanti l'autore del tweet ma anche di terzi. In seguito Aylin et al.\cite{privacyDetective} hanno cercato di capire quanto siano sensibili le informazioni contenute nei tweet attribuendo a questi un \textbf{privacy score}, esso poteva essere di tre tipi:
\begin{enumerate}
    \item il tweet non contiene informazioni private
    \item il tweet può contenere informazioni private
    \item il tweet contiene informazioni private
\end{enumerate}
Questo studio si è concluso con il 10.37\% dei tweet collezionati ha un privacy score di tipo 1, il 21.11\% di tipo e il restante 68.52\% è di tipo 3.
Infine Qiaozhi et al. \cite{dontTweetThis} hanno cercato di fornire un' ulteriore misura della sensibilità presente nei vari tweet raccolti, le misure che hanno fornito sono i \textbf{privscore} essi si dividono in tre categorie
\begin{enumerate}
    \item liberi da contesto, un tweet che contiene un dato sensibile
    \item consapevoli del contesto, a seconda del contesto scritto nel tweet un dato può essere sensibile o meno
    \item personalizzati, a seconda della cronologia dei tweet fatti da un utente un dato può essere sensibile o meno
\end{enumerate}
\end{comment}

\begin{thebibliography}{9}
\bibitem{looseTweets} 
Huina Mao, Xin Shuaiand Apu Kapadia.\newline
\textit{Loose Tweets: An Analysis of Privacy Leaks on Twitter}. \newline
WPES 2011.

\bibitem{privacyDetective}
Aylin Caliskan, Jonathan Walsh, Rachel Greenstadt.\newline
\textit{Privacy Detective: Detecting Private Information and Collective Privacy Behavior in a Large Social Network}.\newline
WPES 2014

\bibitem{dontTweetThis}
Qiaozhi Wang, Hao Xue, Fengjun Li, Dongwon Lee, and Bo Luo.\newline
\textit{\#DontTweetThis: Scoring Private Information in Social Networks}. \newline
POPETS 2019

\bibitem{dataSpectrum}
John M.M. Rumbold, Barbara K. Pierscionek\newline
\textit{What Are Data? A Categorization of the Data Sensitivity Spectrum}.\newline
2018

\bibitem{Random Forest}
L. Breiman, "Random forests",\newline
Mach. Learn., vol. 45, no. 1, pp. 5-32, 2001.

\bibitem{USE}
Cer, Daniel  and Yang, Yinfei  and Kong, Sheng-yi  and
Hua, Nan  and Limtiaco, Nicole  and St. John, Rhomni  and
Constant, Noah  and Guajardo-Cespedes, Mario  and Yuan, Steve  and
Tar, Chris  and Strope, Brian  and Kurzweil, Ray\newline
\textit{Universal Sentence Encoder}, 2018


\end{thebibliography}

\end{document}