\section{Contesto}
Da una decina d'anni a questa parte gli OSN sono sempre più presenti nella vita di milioni di persone, siamo arrivati al punto che una singola persona possiede anche più tipi di social network e su ognuno di questi pubblica contenuti diversi. Questo fa si che su queste piattaforme viaggi una enorme mole di informazioni. A tal proposito è stato dimostrato, che su una sessione di navigazione che comprendeva
un insieme di azioni tipiche di un utente (accesso a social network, ricerche,
posting di commenti, ecc), Google Analytics raccoglieva circa l’87\% dei dati
personali e sensibili disponibili\cite{MalandrinoScarano}

Spesso queste informazioni pubblicate sugli OSN possono contenere informazioni sensibili e/o personali proprie o di terzi. 

Quando gli utenti di un OSN si registrano ad esso questi stanno affidando alla piattaforma alla quale si iscrivono la loro anagrafica e molte altre informazioni che possono contribuire alla identificazione univoca dell'individuo. A gestori degli OSN viene affidato il compito di manutenere e di tenere al sicuro i dati che l'utente gli affida. Purtroppo sono stati riportati dei casi in cui delle falle di sicurezza presenti in alcuni OSN abbiano portato al furto di molti account(e ai dati ad esso connessi).

\subsection{Alcuni casi di furto di informazioni}
\label{ssec:exampleLeak}
\textbf{1. AOL search data leak}\cite{aolDataLeak}\newline
Nel 2006 la compagnia internet AOL pubblica un grande numero di ricerche fatte dagli utenti. Le ricerche pubblicate da AOL erano anonime ma, molte di queste, contenevano informazioni personali sensibili.

Partendo da queste ricerche, il \textit{New York Times} è stato in grado di risalire a molte delle persone che avevano effettuato quelle ricerche.\newline
\textbf{2. Myspace account leakage}\cite{myspacebreach}\newline
Nel 2016 sono stati trovati 360 milioni di account myspace in vendita sul deep web. I dati comprendevamo mail, password, username di una parte degli utenti create prima del giugno 2013.\newline
\textbf{3. Cambridge Analytica}\cite{cambridge}\newline
Nel 2018 viene rivelato che la società di consulenza politica Cambridge Analytica aveva raccolto i dati di milioni di utenti Facebook e li ha usati per scopi di propaganda politica.

I dati raccolti da Cambridge Analytica sono stati utilizzati per la campagna elettorale di Donald Trump, Brexit, e le elezioni messicane del 2018.\newline
\textbf{4. Caso Renzi Formigli}\cite{formigli}\newline
In questo caso non c'è un colosso della IT che per una falla di sicurezza perde le informazioni sensibili dei sui utenti ma una persona che rivela in maniera indiretta dati sensibili di una terza.

Il giornalista e conduttore televisiovo Corrado Formigli intervista il senatore della Repubblica Matteo Renzi in merito al prestito che gli era stato fatto da un suo amico per l’acquisto di una villa.


Durante la trasmissione il conduttore mostra le foto della villa, poche ore dopo il termine della trasmissione su Facebook compaiono post che parlano e mostrano la casa del giornalista.

\section{Privacy leakage}
\label{sec:privacyLeakage}
Nella sezione \ref{ssec:exampleLeak} sono stati illustrati alcuni esempi di \textbf{privacy leakage}, ovvero, contenuti sensibili o inappropriati che gli utenti stessi, inconsapevolmente o involontariamente, divulgano durante l'utilizzo dei social o altri servizi (es. e-mail).I contenuti possono riguardare la propria vita privata o quella altrui.

\section{Norme per tutelare la privacy degli utenti}
Per migliorare e regolamentare la gestione dei dati sensibili e personali sul web la Commissione Europea ha iniziato ad aggiornare le regole in ambito privacy dal 2016, concentrandosi sul controllo della diffusione dei dati e sul conoscere da \textit{chi} e \textit{come}, questi vengano trattati.

La normativa europea GDPR(General Data Protection Regulation)\footnote{https://gdpr-info.eu/} definisce come \textit{dati personali} le informazioni che identificano o rendono identificabile, direttamente o indirettamente, una persona fisica e che possono fornire informazioni sulle sue caratteristiche, le sue abitudini, il suo stile di vita le sue relazioni personali, il suo stato di salute, la sua situazione economica, ecc.

Il GDPR quindi riduce il rischio di uso improprio e violazioni della privacy,
causate intenzionalmente o per mancanza di consapevolezza.

Inoltre la Convenzione 108\footnote{https://www.coe.int/en/web/data-protection/convention108/modernised} e il Regolamento Europeo prevedono che i dati devono
essere conservati per un periodo di tempo limitato, e in particolare non
oltre il tempo necessario per raggiungere lo scopo alla base del trattamento.
Nel caso in cui un titolare del trattamento volesse mantenerli per un periodo
superiore, deve procedere alla loro anonimizzazione.

\section{Cos'è Knoxly}
\label{sec:whoisKnoxly}
Knoxly (Figura \ref{fig:logoKnoxly}) si prefigge come obiettivo quello di aiutare gli utenti ad avere una maggiore consapevolezza
dei dati che pubblicano/condividono sul Web , dato che il principale rischio per la privacy online è la divulgazione di
informazioni personali e sensibili degli stessi utenti.\newline
\begin{figure}[h]
    \centering
    \includegraphics[scale=0.25]{Figure/logoKnoxly.png}
    \caption{logo di Knoxly}
    \label{fig:logoKnoxly}
\end{figure}
\FloatBarrier
Spesso gli utenti degli OSN divulgano anche senza la piena consapevolezza dati sensibili e/o personali proprio o di terzi. A tal fine, Knoxly (1) individua quali dati sensibili e personali sono presenti in un messaggio di testo che l’utente è intenzionato a pubblicare/condividere sul Web, (2) li mostra attraverso un’ interfaccia grafica user-friendly e (3) lascia all’utente decidere quali o quante informazioni sensibili e personali è disposto a condividere con la divulgazione di tale messaggio.

A differenza dei convenzionali meccanismi di protezione della privacy sui dati o sui OSN, che si focalizzano principalmente sulla protezione dell’identità o di attributi privati \cite{collective-data, diff-privacy, urwho, inferr-privacy, protection-private, stalking}, Knoxly si muove verso l’idea di \quotes{\textit{privacy as having the ability to control the dissemination of sensitive information}}.

Attualmente i limiti di Knoxly sono quelli intrinsechi ad una analisi lessicale. Per questa ragione, in questo lavoro verrà aggiunto un modulo semantico in grado di capire se un testo scritto da un utente contenga o meno dati sensibili e/o personali.

\section{Il modello di minaccia}
Knoxly mira a proteggere gli utenti del Web dalla diffusione accidentale di qualsiasi contenuto inappropriato, in particolare le informazioni private o sensibili su se stessi. Si considera principalmente il rischio di una diffusione inappropriata a due tipo di pubblico: (1) follower o amici, che ricevono aggiornamenti dei post dell'utente; (2) staker esterni, che sbirciano nei post dei social network di un utente target. È probabile che entrambi conoscano l'identità offline dell'utente. Knoxly non impedisce all'utente di pubblicare i contenuti(sensibili) nè impedisce al destinatario di visualizzare i contenuti, fornendo privacy awareness. L'idea di Knoxly parte dal presupposto che dei tracciatori possono esplorare l'OSN attraverso l'interfaccia utente o raccogliere dati utilizzando un crawler automatizzato tramite l'API dell'OSN di riferimento. Infine, non considera la retrazione/cancellazione dei post precedenti, nè li analizza.

\section{Obiettivi del lavoro}
Questo lavoro si prefigge l'obiettivo di aggiungere a Knoxly una intelligenza artificiale(IA) in grado di poter riconoscere un dato sensibile, di segnalare all'utente la presenza di questo dato e di dare la possibilità all'utente di inviare un feedback all'IA in modo tale che essa possa imparare a riconoscere quali sono i dati sensibili secondo l'utente che sta utilizzando il plugin.
\subsection{I dati sensibili}
\label{ssec:sensitive_data}
Secondo la normativa GDPR attualmente in vigore per dati sensibili si intendono i dati che rivelino:
\begin{itemize}
    \item l'etnia, pensiero politico, religioso o filosofico
    \item se si è membri di un sindacato
    \item dati genetici o biomedici che permettano l'identificazione univoca di un essere umano
    \item dati relativi alla salute
    \item orientamento sessuale e vita sessuale
\end{itemize}
Abbiamo scelto di utilizzare un sottoinsieme di temi di partenza per dare una conoscenza di base all'IA di Knoxly. L'IA inizialmente sarà in grado di individuare dati sensibili riguardo i seguenti topic
\begin{enumerate}
    \item orientamento politico
    \item salute
    \item lavoro
    \item viaggi
\end{enumerate}
Questi temi sono stati scelti per due motivi, il primo è che molti di questi temi fanno parte della normativa GDPR quindi è una legge a dirci che i dati appartenenti a quelle categorie sono sensibili e il secondo è che molti dei temi che verranno trattati sono già stati utilizzati negli studi sopracitati ma anche in altri studi riguardanti la privacy detection e il privacy leakage.